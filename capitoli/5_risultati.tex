\chapter{Risultati}
\section{Differenze tra modello normale e compresso}
Dati i problemi che comportavano l'utilizzo dei modelli da 4.8 GB, si è scelto di utilizzare per l'analisi dei risultati le versioni compresse da appena 20 MB ottenute tramite product quantization.
Per misurare la differenza di prestazioni tra il modello originale e la sua versione compressa si è considerato come riferimento il modello di Biijeta et alii, avente le seguenti caratteristiche:
\begin{itemize}
    \item traduzione della sequenza dei tasti premuti con \texttt{word2keypress}
    \item numero minimo di n-gram pari a 1;
    \item numero di epoche di training pari a 5.
\end{itemize}

Per entrambi i modelli si è tenuto conto del valore di precision e recall, in modo da fornire una valutazione efficace del modello.
Sono state osservate differenze minime riguardanti i valori, motivo per il quale si è scelto di considerare soltanto le versioni compresse per valutare gli altri modelli.

\section{Euristiche adottate}
\subsection{Ground truth e prediction}
Ground truth: cosa è
Prediction: cosa è
euristica.

\section{Similarità: un confronto}
\section{Criticità del modello proposto da Bijeeta et alii}
\section{Rappresentazione grafica della distanza tra parole}

\label{ch:risultati}

