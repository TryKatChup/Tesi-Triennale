% Rappresenta il tipo di documento
\documentclass[12pt, twoside]{book}

% Elenco dei packages utilizzati
\usepackage[a4paper,width=150mm,top=25mm,bottom=25mm,bindingoffset=6mm]{geometry}
\usepackage[utf8]{inputenc}
\usepackage[T1]{fontenc}
\usepackage[italian]{babel}
\usepackage{subcaption}
\usepackage{graphicx}
\usepackage{fancyhdr}
\usepackage{float}
\usepackage{color}
\usepackage[nottoc]{tocbibind}
\usepackage{emptypage}
\usepackage[hidelinks]{hyperref}
\usepackage{titlesec}
\usepackage{amssymb}
\usepackage[sorting=none,backend=bibtex]{biblatex}
\usepackage{csquotes}
\usepackage{pdfpages}
\usepackage{amsmath}
\usepackage{placeins}
\usepackage{hyperref}
\usepackage{footnote}
\usepackage{minted}
\usepackage{xcolor}

% Stile di pagina
\pagestyle{fancy}
\definecolor{LightGray}{gray}{0.9}
\usemintedstyle{colorful}

\renewcommand{\chaptermark}[1]{\markboth{#1}{}}
\renewcommand{\tocetcmark}[1]{\markboth{#1}{}}
\renewcommand{\sectionmark}[1]{\markright{\thesection\ #1}}
\fancyhf{}
\fancyhead[LE,RO]{\bfseries\thepage}
\fancyhead[LO]{\bfseries\rightmark}
\fancyhead[RE]{\bfseries\leftmark}
\renewcommand{\headrulewidth}{0.5pt}
\fancypagestyle{plain}{
    \fancyhead{}
    \renewcommand{\headrulewidth}{0pt}
}
\setlength{\headheight}{28pt}

% Stile dei titoli dei capitoli
\definecolor{gray75}{gray}{0.75}
\newcommand{\hsp}{\hspace{20pt}}
\titleformat{\chapter}[hang]
{\Huge\bfseries}
{\thechapter\hsp\textcolor{gray75}{|}\hsp}
{0pt}
{\Huge\bfseries}

% Impostazioni bibliografia
\addbibresource{./bibliografia/bibliografia.bib}

% Ridefinizione enfasi
%\DeclareTextFontCommand{\emph}{\bfseries\em}

% Inizio del documento
\begin{document}

% Frontespizio
\includepdf{./frontespizio/frontespizio.pdf}

\cleardoublepage
\begin{flushright}
\thispagestyle{empty}
\null\vspace{\stretch {1}}
\textit{
    “Arguing that you don't care about the right to privacy\break because you have nothing to hide is no different than saying\break you don't care about free speech because you have nothing to say.”
    \break --- Edward Snowden.}
\vspace{\stretch{2}}\null
\end{flushright}
\cleardoublepage

% Indice
\tableofcontents

% Elenco delle figure
\phantomsection
\listoffigures

% Introduzione
\chapter*{Introduzione}
\chaptermark{Introduzione}
\addcontentsline{toc}{chapter}{Introduzione}
La sicurezza informatica risulta importante per la salvaguardia dei dati personali. In particolare, è necessario assicurare i seguenti due aspetti:
\begin{itemize}
    \item \textbf{Confidenzialità:} la garanzia per la quale un utente non autorizzato non possa accedere a informazione riservate.
    \item \textbf{Integrità:} la garanzia per la quale un utente non autorizzato non possa modificare informazioni e dati personali.
\end{itemize}
Esistono diversi meccanismi di autenticazione, divisi tra:
\begin{itemize}
    \item ``Cosa si possiede'': oggetti fisici fini all'autenticazione, tra i quali chiavi, badge elettronici, token fisici, ecc.
    \item ``Cosa si è'': identificazione tramite caratteristiche fisiche, tra cui il riconoscimento facciale, l'impronta digitale, scansione dell'iride.
    \item ``Cosa si sa'': sequenze alfanumeriche che compongono una password o un PIN, informazioni personali private.
\end{itemize}
In questo documento si pone maggiore attenzione alla generazione di parole chiave, poiché risulta il meccanismo maggiormente utilizzato e facilmente raggirabile. Esistono diverse tecniche di attacco che compromettono la sicurezza che una password solitamente offre, come ad esempio gli attacchi a forza bruta che tentano tutte le possibili combinazioni di caratteri, oppure gli attacchi a dizionario che utilizzano un corpus di parole.

Le strategie di attacco possono essere di due tipologie in base a se si possiedono dati riguardanti un determinato utente: mirato, nel caso in cui si disponga di informazioni personali riguardanti l'utente e ottenute tramite ingegneria sociale o leak di dati, non mirato altrimenti.

In seguito alla creazione di molteplici profili da parte di un utente e violazioni di dati personali è spesso riscontrata la tendenza di riutilizzare la stessa password ~\cite{google} o varianti di essa. Ciò può comportare la compromissione della propria identità e pericoli relativi alla privacy online.

In questo lavoro di tesi vengono analizzati i pericoli relativi al fenomeno di password reuse, e proposti approcci di prevenzione, che sfruttano tecniche di machine learning basati su lavori presenti in letteratura~\cite{bijeeta}.

Si cercherà in particolare di approfondire l'approccio utilizzato, in modo da potere proporre migliori strategie fini alla analisi della similarità tra password.

% Capitoli
\chapter{Stato dell'arte}
\label{ch:state of the art}

La sicurezza delle password al giorno d'oggi riveste un ruolo significativo nel garantire confidenzialità e integrità dei dati personali degli utenti e delle aziende.
Solitamente si è più propensi a scegliere password semplici da ricordare, come riferimenti autobiografici, oppure sequenze di caratteri molto comuni (e.g. \texttt{qwerty}, \texttt{123456}).
Per semplificare la memorizzazione, si utilizzano spesso password brevi, in media da 9-10 caratteri e composte in gran parte da caratteri minuscoli~\cite{obspasshab}.

Tuttavia questa scelta porta a maggiori probabilità di subire violazioni dei propri account, poiché password semplici sono vulnerabili ad attacchi di forza bruta. Inoltre, tramite tecniche di ingegneria sociale, è possibile individuare il criterio di scelta dell'utente, eventualmente ragionando sui dati disponibili grazie ai data breach.

Nel 2020 sono stati confermati 3950 data breach, dal costo medio di 3,86 milioni di dollari. Il 52\% dei breach è stato causato da attacchi informatici e il numero di giorni medio per individuare un breach è stato di 207 giorni~\cite{ibmcost}.
Il 42\% è causato da attacchi su applicazioni web e il metodo più comune di attacco (82\%) ha utilizzato credenziali rubate o ottenute tramite attacchi a forza bruta.
Il 58\% dei breach conteneva dati personali~\cite{verizon}.
\begin{figure}[h]
    \centering
    \includegraphics[width=15cm]{./immagini/prova.png}
    \label{data breach}
    \caption{Costo medio e frequenza di data breach causati da attacchi informatici, in base alla causa, nel 2020~\cite{ibmcost}}
\end{figure}

Sebbene la maggior parte delle password siano crittografate, è possibile risalire alla forma testuale mediante strumenti come Hashcat\footnote{\url{https://github.com/hashcat/hashcat}} e John The Ripper\footnote{\url{https://github.com/openwall/john}}.

In seguito alla diffusione delle credenziali, gli utenti decidono di cambiare password e la scelta ricade spesso su varianti usate su altri account.
\section{Scelta di una nuova password}
\label{sec:scelta di una nuova password}
L'utente medio ha la tendenza a scegliere password semplici~\cite{obspasshab}. Per questo motivo, spesso la nuova password è il risultato di una leggera variazione della vecchia password, o una combinazione di password precedenti~\cite{hypr}.

Un modo per verificare la sicurezza della password è utilizzare strumenti come zxcvbn\footnote{\url{https://github.com/dropbox/zxcvbn}}, che riesce a riconoscere:
\begin{itemize}
    \item 30000 password comuni;
    \item nomi e cognomi comuni negli USA;
    \item parole spesso utilizzate in inglese su Wikipedia;
    \item parole spesso utilizzate alla televisione e film statunitensi;
    \item date;
    \item ripetizioni di lettere (\texttt{aaaa});
    \item sequenze alfabetiche (\texttt{abcde});
    \item sequenze di tastiera (\texttt{qwertyuiop});
    \item il codice \texttt{l33t}.
\end{itemize}

Altri strumenti, come Kaspersky password checker\footnote{\url{https://password.kaspersky.com/it/}}, controllano anche dati di numerosi data breach raccolti da Have I been Pwned?\footnote{\url{https://haveibeenpwned.com/}}.
Questi approcci, tuttavia, non controllano la cronologia delle password di uno specifico utente, ma soltanto la resistenza ad attacchi di forza bruta.

Per questo motivo sono state studiate strategie che tengono conto delle credenziali utilizzate.
Alcune sfruttano un approccio probabilistico, come i \emph{Bloom filter}~\cite{bloom}. Un'altra possibile modalità utilizza \emph{word embedding} di password.

\section{Word embedding}
\label{sec:word embedding}
Per capire il contesto delle parole e per poterle rappresentare in base alla sfera semantica e alla sintassi, si ricorre a un insieme di tecniche che prevedono il mapping delle parole o delle frasi di un dizionario in vettori di numeri reali, note come \emph{word embedding}.
Parole simili possiedono una codifica simile.


Per stabilire il valore di ogni embedding si allena una rete neurale con specifici parametri e le dimensioni variano tra 8 (per piccoli dataset) a 1024. Maggiore è la dimensione di un embedding, maggiore risulta la quantità di informazioni relativa alle relazioni tra parole~\cite{tensword}.

\subsection{Similarità tra parole}
\label{Similarita tra parole}
Per potere stabilire se due parole appartengono alla stessa sfera semantica si utilizza un metodo noto come \emph{cosine similarity}.
\begin{equation}
    similarity = cos(\theta) = \frac{A\cdot{B}}{\|A\|\cdot{\|B\|}}
\end{equation}

Due parole risultano simili quando il valore del coseno è 1, ovvero quando l'angolo tra i due vettori risulta nullo. Si consideri il seguente esempio:

\begin{figure}[h]
    \centering
    \includegraphics[width=11cm]{./immagini/cosine_similarity_esempio.png}
    \label{cosine}
    \caption{Esempio di calcolo della similarità tra word embedding~\cite{cosine}}
\end{figure}

Nella figura sono mostrati i vettori di 3 parole (\textit{cherry}, \textit{digital} e \textit{information}) in uno spazio bidimensionale definiti dal numero di occorrenze in vicinanza alle parole \textit{computer} e \textit{pie}.
Come si può notare, l'angolo tra \textit{digital} e \textit{information} risulta minore rispetto all'angolo tra \textit{cherry} e \textit{information}.
Quando due vettori risultano più simili tra loro, il valore del coseno risulta maggiore, ma l'angolo risulta minore. Il coseno assume valore massimo 1 quando l'angolo tra i due vettori risulta nullo (0°); il coseno degli altri angoli risulta inferiore a 1~\cite{cosine}.
\subsection{Tipologie di word embedding}
\label{sec:tipologie di word embedding}
\subsubsection{Word2Vec}
\label{sec:word2vec}
Word2Vec è un insieme di modelli architetturali e di ottimizzazione utilizzati per imparare word embedding da un vasto corpus di dati, sfruttando reti neurali.
\\
Un modello allenato con Word2Vec riesce a individuare le parole simili tra loro, in base al contesto, grazie alla \textit{cosine similarity} esaminata precededentemente.

\begin{figure}[h]
    \centering
    \includegraphics[width=15cm]{./immagini/cbow_vs_skipgram.png}
    \label{cbowskipgram}
    \caption{CBOW vs skip-gram~\cite{mikolov2013efficient}}
\end{figure}
\newpage
Word2Vec utilizza due modelli di architetture:
\begin{itemize}
    \item \textbf{CBOW} (continuous bag of words): l'obiettivo del training è combinare le rappresentazioni delle parole limitrofe per prevedere la parola centrale.
    \item \textbf{Skip-gram}: simile a CBOW, con la differenza che viene utilizzata la parola centrale per prevedere le parole circostanti relative allo stesso contesto\cite{fasttext}.
\end{itemize}

CBOW risulta più veloce ed efficace in caso di dataset di grandi dimensioni, tuttavia, nonostante la maggiore complessità, Skip-gram è in grado di trovare parole non presenti nel corpus, per dataset di minori dimensioni~\cite{mikolov2013efficient}.


\subsubsection{FastText}
\label{sec:fasttext}
FastText è una libreria open-source proposta da Facebook che estende Word2Vec, e consente un apprendimento efficiente di rappresentazioni di parole e di classificazioni di frasi.
Anziché allenare un modello fornendo ogni singola parola di un dataset, FastText prevede l'apprendimento tramite \textit{n-gram} di ciascuna parola.

Si definiscono n-gram di una parola costituita da $c_1...c_m$ caratteri la seguente sequenza:
\begin{center}
$\{ c_{i_1}, c_{i_2}, \ldots, c_{i_n} \mid \sum\limits_{j=1}^n i_{j} - i_{j-1} < 0 \}$
\end{center}
Ad esempio, gli n-gram della parola ciao, con $n\_mingram = 1$ e $n\_maxgram = 4$, sono i seguenti:
%Si considerino gli n-gram della parola $ciao$:

\begin{center}
    $ciao = \{\{c, i, a, o\},\{ci, ia, ao\}, \{cia, iao\}, \{ciao\}\}$
\end{center}
$ciao$ viene espresso come l'insieme di tutte le sottostringhe di lunghezza minima pari a 1, e lunghezza massima pari a 4.

%In questo esempio, con $n\_mingram = 1$ e $n\_maxgram = 4$, $ciao$ viene espresso come l'insieme di tutte le sottostringhe di lunghezza minima pari a 1, e lunghezza massima pari a 4.

FastText consente di ottenere, con più probabilità rispetto a Word2Vec, parole \emph{out-of-dictionary}, ovvero parole sconosciute al modello in fase di training.

\chapter{Analisi progettuale}
\label{ch:analisi progettuale}

\chapter{Implementazione}
\label{ch:implementazione}
\section{Allenamento di un modello di word embedding tramite FastText}
\subsection{Introduzione}
Lo scopo del progetto è sviluppare un classificatore che, date due password, determini se sono simili tra loro (e quindi potenzialmente non sicure) o meno.

A tal proposito sono stati implementati modelli di word embedding, i quali permettono di rappresentare parole in uno spazio vettoriale preservando proprietà semantiche quali la similarità e la co-occorrenza nel medesimo contesto.
Come modello è stato scelto FastText, il quale sfrutta informazioni sugli n-gram della parola per determinarne l'embedding~\cite{biijeta}.

Si è deciso di allenare 5 diversi modelli:
\begin{itemize}
    \item Il modello di Bijeeta et alii~\cite{biijeta} che utilizza la libreria \texttt{word2keypress} per ricavare la sequenza di tasti premuti per specifici caratteri, numero minimo di n-gram pari a 1, e numero di epoche per allenare la rete pari a 5.
    \item Un modello che utilizza \texttt{word2keypress}, con numero minimo di n-gram pari a 2, e numero di epoche per allenare la rete pari a 10.
    \item Tre modelli che non rilevano la sequenza di tasti premuti:
    \begin{itemize}
        \item Uno con numero minimo di n-gram pari a 1, e numero di epoche per allenare la rete pari a 5.
        \item Uno con numero minimo di n-gram pari a 2, e numero di epoche per allenare la rete pari a 5.
        \item Uno con numero minimo di n-gram pari a 2, e numero di epoche per allenare la rete pari a 10.
    \end{itemize}
\end{itemize}

\subsection{Precondizioni}
Prima della fase di allenamento della rete, è necessario avere un dataset valido di password; nel progetto è stato utilizzato lo stesso data breach da 45 GB citato da Bijeeta et alii~\cite{biijeta}, tuttavia sono stati scelti criteri diversi di filtraggio:
\begin{itemize}
    \item Sono state rimossi gli account con password con lunghezza inferiore a 4 caratteri o maggiore di 30.
    \item Sono state rimosse gli account con password che presentavano caratteri non ASCII o non stampabili.
    \item Sono stati rimossi gli account creati da bot, riconoscibili grazie al numero di occorrenze della stessa email nel dataset, superiore a 100.
    \item Sono stati rimossi gli account con password contententi sequenze esadecimali (identificate da \texttt{\$HEX[]} e \texttt{\textbackslash x}).
    \item Sono stati rimossi sequenze che rappresentano caratteri in HTML, come:
    \begin{itemize}
        \item \texttt{\&gt} (simbolo $>$);
        \item \texttt{\&ge} (simbolo $\geq$);
        \item \texttt{\&lt} (simbolo $<$)
        \item \texttt{\&le} (simbolo $\leq$);
        \item \texttt{\&\#} (ovvero i codici di entità in HTML);
        \item \texttt{amp}.
    \end{itemize}
    \item Sono stati rimossi gli account che presentavano meno di 2 password, poiché più password per utente risultano indispensabili per ricavare la similarità.
\end{itemize}
Successivamente, in accordo con Bijeeta et alii~\cite{biijeta}, per i modelli che prevedevano la memorizzazione delle sequenze di tasti premuti, è stata utilizzata la libreria python \href{https://github.com/rchatterjee/word2keypress}{\texttt{word2keypress}}.
Successivamente i risultati sono stati salvati in un file \texttt{csv} nel seguente formato:
\begin{center}
    \texttt{sample@gmail.com: ["'97314348'", "'voyager<s>1'"]}
\end{center}

\subsection{Impostazioni dell'ambiente}
Per potere allenare il modello è stato utilizzato \texttt{gensim.FastText}~\cite{gensim}.
\\
\texttt{gensim} è una libreria python multipiattaforma e open source che racchiude un'ampia scelta di modelli di word embedding pre allenati~\cite{gensim}.
Per estrapolare i dati dal file \texttt{csv} è stata sviluppata una classe ausiliaria \texttt{PasswordRetriever}.
\subsection{Allenamento del modello}
\subsubsection{Parametri di FastText}
\begin{minted}[
    frame=lines,
    framesep=2mm,
    baselinestretch=1.2,
    bgcolor=LightGray,
    fontsize=\footnotesize,
    linenos
    ]{python}
negative = 5
subsampling = 1e-3
min_count = 10
min_n = 2
max_n = 4
SIZE = 200
sg = 1
\end{minted}
Per questo progetto è stato utilizzato il modello \emph{Skip-gram} (\texttt{sg = 1}) e il negative sampling~\cite{negative}:
\begin{itemize}
    \item Il modello Skip-gram (parametro \texttt{sg = 1}) è stato scelto in quanto la rappresentazione distribuita dell'input è stata utilizzata per prevedere il contesto delle password. In particolare, risulta efficace per determinare quali caratteri intorno a una specifica sequenza sono presenti; in questo modo il modello riesce a imparare password non presenti nel corpus fornito per l'allenamento.~\cite{fasttext}
    \item Il negative sampling (parametro \texttt{negative = 5}) rende l'allenamento più veloce, poiché ciascuna sezione dell'allenamento aggiorna solo una piccola percentuale dei pesi del modello.~\cite{negative} Per dataset di dimensioni maggiori (come in questo caso) è consigliabile impostare il suo valore tra 2 e 5.~\cite{gensim}
    \item La dimensione del vettore contenente gli embedding è impostato a $200$ (parametro \texttt{SIZE = 200}), in modo da potere allenare più velocemente il modello. Normalmente viene raccomandata una dimensione pari a \texttt{SIZE = 300}.~\cite{gensim}
    \item Il subsampling ignora le password più frequenti, ovvero che presentano più di 1000 occorrenze.
    \item \texttt{mincount} rappresenta il numero minimo di occorrenze di una password nel dataset di training affinché venga considerata nell'allenamento~\cite{biijeta}~\cite{gensim}.
    \item \texttt{min\_n} e \texttt{max\_n} rappresentano rispettivamente il numero minimo e massimo degli n-gram. Gli N-gram vengono utilizzati per prevedere una sequenza di caratteri e il contesto di quest'ultima. In questo caso essi rappresentano una sequenza di caratteri contigui e il loro scopo è di dare informazioni di contesto e posizionali di una determinata sequenza all'interno di una password~\cite{biijeta}~\cite{gensim}.
\end{itemize}

\subsubsection{Allenare FastText}
La lista di password relativa a ciascun utente (\texttt{password\_list}) viene ottenuta grazie alla classe ausiliaria \texttt{PasswordRetriever}. Il modello di FastText si basa su \texttt{password\_list} e viene allenato con i parametri elencati precedentemente.

\begin{minted}[
    frame=lines,
    framesep=2mm,
    baselinestretch=1.2,
    bgcolor=LightGray,
    fontsize=\footnotesize,
    linenos
    ]{python}
filename='../train.csv'
password_list = PasswordRetriever(filename)
trained_model = FastText(password_list, size=SIZE, min_count=min_count,
                        workers=12, negative=negative,
                        sample=subsampling, window=20,
                        min_n=min_n, max_n=max_n)
\end{minted}

\subsection{Compressione del modello}
Il modello allenato ha un peso complessivo pari a 4.8 GB, e ciò comporta i seguenti problemi:
\begin{itemize}
    \item Meno efficiente in termini di spazio, di conseguenza l'utilizzo del modello è limitato su sistemi con vincoli di memoria o di spazio.
    \item È difficile utilizzare il modello in contesti distribuiti, poiché non è facilmente trasportabile.
\end{itemize}
Per comprimere il modello si è utilizzata la libreria \texttt{compress\_fasttext}, che sfrutta tecniche di quantizzazione e di feature selection.~\cite{compress-fasttext}
\begin{minted}[
    frame=lines,
    framesep=2mm,
    baselinestretch=1.2,
    bgcolor=LightGray,
    fontsize=\footnotesize,
    linenos
    ]{python}
big_model = gensim.models.fasttext.load_facebook_vectors('model.bin')
small_model = compress_fasttext.prune_ft_freq(big_model, pq=True)
small_model.save('compressed_model')
\end{minted}

Viene definito come \emph{feature selection} il processo di selezione delle feature più importanti da usare per costruire un modello~\cite{feature}.

Per \emph{Product quantization} si intende un particolare tipo di quantizzazione vettoriale, che viene utilizzata per la compressione di modelli di linguaggio naturale e di elaborazione di immagini e consente di generare in modo non esponenziale una quantità grande di codice in tempi contenuti e con costi ridotti in termini di memoria~\cite{biijeta}~\cite{compress-fasttext}~\cite{quantization}.

Il modello compresso ottenuto dalle operazioni di quantizzazione e di feature selection ha una dimensione di 20 MB.
\chapter{Risultati}
\label{ch:risultati}
\section{Differenze tra modello normale e compresso}
\label{sec:differenze modello normale e compresso}
Dati i problemi che comportavano l'utilizzo dei modelli da 4.8 GB, si è scelto di utilizzare per l'analisi dei risultati le versioni compresse da appena 20 MB ottenute tramite product quantization.
Per misurare la differenza di prestazioni tra il modello originale e la sua versione compressa si è considerato come riferimento il modello di Bijeeta et alii, avente le seguenti caratteristiche:
\begin{itemize}
    \item traduzione della sequenza dei tasti premuti con \texttt{word2keypress}
    \item numero minimo di n-gram pari a 1;
    \item numero di epoche di training pari a 5.
\end{itemize}
Per entrambi i modelli si è tenuto conto del valore di precision e recall, in modo da fornire una valutazione efficace del modello.
Non sono state osservate differenze significative riguardanti i valori, motivo per il quale si è scelto di considerare soltanto le versioni compresse per valutare gli altri modelli.

\begin{figure}[H]
    \centering
    \includegraphics[width=11.5cm]{./immagini/big_model.png}
    \caption{Precision e recall nel modello non compresso di Bijeeta et al.~\cite{bijeeta} con le euristiche definite nel paragrafo \ref{sec:risultati euristiche adottate}, con \texttt{word2keypress}, \texttt{n\_mingram = 1},\\\texttt{epoche = 5}}
    \label{bigmodel}
\end{figure}

\begin{figure}[H]
    \centering
    \includegraphics[width=11.5cm]{./immagini/CORRETTO_w2kp_nmingram=1_epochs=5.png}
    \caption{Precision e recall nel modello compresso di Bijeeta et al.~\cite{bijeeta} con le euristiche definite nel paragrafo \ref{sec:risultati euristiche adottate}, con \texttt{word2keypress}, \texttt{n\_mingram = 1},\\\texttt{epoche = 5}}
    \label{primomodello}
\end{figure}

\section{Classificazione del modello}
\label{sec:classificazione modello}
\subsection{Euristiche adottate}
\label{sec:risultati euristiche adottate}
Per valutare i modelli, a differenza di Bijeeta et alii~\cite{bijeeta}, non si è utilizzato Pass2Path a causa della complessa implementazione. Si è invece adottata la seguente euristica:
\begin{itemize}
    \item verifica della password con la variante minuscola;
    \item verifica della password con la variante maiuscola;
    \item verifica della password con la traduzione in codice \texttt{l33t};
    \item verifica se l'edit distance della password supera 5.
\end{itemize}

\subsection{Ground truth e prediction}
\label{sec:ground truth e prediction}
Per \emph{ground truth} si intende il risultato ideale che ci si aspetta e viene utilizzato per verificare una correttezza delle previsioni del modello.~\cite{lemoigne2008molecular}

Il termine \emph{prediction} si riferisce all'output di un modello dopo che è stato allenato su un dataset e applicato a nuovi dati quando si vuole prevedere la probabilità di un certo evento.~\cite{prediction}.

Per calcolare il valore di ground truth si tiene conto delle candidate due password e della euristica scelta; per ottenere il valore di prediction invece si considera si utilizza la funzione \texttt{gensim.similarity} del modello allenato, che sfrutta la cosine similarity discussa nel capitolo \ref{Similarita tra parole}.
Nel caso in cui si consideri la variante con \texttt{word2keypress} occorre convertire le due password in sequenza di tasti premuti prima di ricavare il valore di prediction.
\subsection{Precision e recall}
Per ricavare precision e recall si utilizza l'approccio di cui accennato nel paragrafo \ref{sec:classificazione password}; a questo scopo si definiscono i seguenti parametri:
\begin{itemize}
    \item \textbf{Veri positivi (TP)}: viene incrementato se sia la prediction che la ground truth hanno valore positivo non nullo;
    \item \textbf{Falsi positivi (FP)}: viene incrementato se la ground truth è nulla e la similarità è positiva non nulla;
    \item \textbf{Falsi negativi (FN)}: viene incrementato se la ground truth è positiva non nulla e la similarità è nulla;
\end{itemize}
Successivamente si ricava il valore di precision e recall nel seguente modo (cfr. paragrafo \ref{sec:classificazione password}):
\begin{gather*}
precision = \frac{TP}{TP + FP}
\\
recall = \frac{TP}{TP + FN}
\end{gather*}

\section{Similarità: un confronto}
\label{sec:similarita, confronto tra modelli}
In base a quanto spiegato nel paragrafo \ref{sec:classificazione password} si sono ottenuti i seguenti grafici:
\begin{figure}[H]
    \centering
    \includegraphics[width=14cm]{./immagini/CORRETTO_w2kp_nmingram=1_epochs=5.png}
    \caption{Precision e recall nel modello di Bijeeta et al.~\cite{bijeeta} con le euristiche definite nel paragrafo \ref{sec:risultati euristiche adottate}, con \texttt{word2keypress}, \texttt{n\_mingram = 1}, \texttt{epoche = 5}}
    \label{primomodello}
\end{figure}
\begin{figure}[H]
    \centering
    \includegraphics[width=14cm]{./immagini/no_w2kp_nmingram=1_epochs=5.png}
    \caption{Precision e recall nel modello senza \texttt{word2keypress}, \texttt{n\_mingram = 1}, \texttt{epoche = 5}}
    \label{secondomodello}
\end{figure}

\begin{figure}[H]
    \centering
    \includegraphics[width=14cm]{./immagini/CORRETTO_w2kp_nmingram=2_epochs=10.png}
    \caption{Precision e recall nel modello con \texttt{word2keypress}, \texttt{n\_mingram = 2}, \texttt{epoche = 10}}
    \label{terzomodello}
\end{figure}

\begin{figure}[H]
    \centering
    \includegraphics[width=14cm]{./immagini/no_w2kp_nmingram=2_epochs=10.png}
    \caption{Precision e recall nel modello senza \texttt{word2keypress}, \texttt{n\_mingram = 2}, \texttt{epoche = 10}}
    \label{quartomodello}
\end{figure}

\begin{figure}[H]
    \centering
    \includegraphics[width=14cm]{./immagini/no_w2kp_nmingram=2_epochs=5.png}
    \caption{Precision e recall nel modello senza \texttt{word2keypress}, \texttt{n\_mingram = 2}, \texttt{epoche = 5}}
    \label{quintomodello}
\end{figure}

\section{Risultati ottenuti}
\label{sec:risultati ottenuti}
Come accennato in \ref{sec:classificazione password}, per classificare le password è necessario definire una soglia $\alpha$ di similarità, in modo da definire due password simili tra loro se la loro similarità supera $\alpha$, diverse tra loro altrimenti.
Analizzando i modelli ottenuti, si possono osservare i seguenti risultati:
\begin{itemize}
    \item con \texttt{word2keypress}, \texttt{n\_mingram = 1}, \texttt{epoche = 5} (modello di Bijeeta et
    \\alii~\cite{bijeeta}, figura \ref{primomodello}):
    \begin{itemize}
        \item $\alpha = 0.5$: precision pari a $\simeq 57\%$ e recall pari a $\simeq 95\%$
        \item $\alpha = 0.6$: precision pari a $\simeq 67\%$ e recall pari a $\simeq 89\%$
    \end{itemize}
    \item senza \texttt{word2keypress}, \texttt{n\_mingram = 1}, \texttt{epoche = 5} (figura \ref{secondomodello}):
    \begin{itemize}
        \item $\alpha = 0.5$: precision pari a $\simeq 60\%$ e recall pari a $\simeq 95\%$
        \item $\alpha = 0.6$: precision pari a $\simeq 73\%$ e recall pari a $\simeq 88\%$
    \end{itemize}
    \item con \texttt{word2keypress}, \texttt{n\_mingram = 2}, \texttt{epoche = 10} (figura \ref{terzomodello}):
    \begin{itemize}
        \item $\alpha = 0.5$: precision pari a $\simeq 62\%$ e recall pari a $\simeq 94\%$
        \item $\alpha = 0.6$: precision pari a $\simeq 73\%$ e recall pari a $\simeq 87\%$
    \end{itemize}
    \item senza \texttt{word2keypress}, \texttt{n\_mingram = 2}, \texttt{epoche = 10} (figura \ref{quartomodello}):
    \begin{itemize}
        \item $\alpha = 0.5$: precision pari a $\simeq 65\%$ e recall pari a $\simeq 94\%$
        \item $\alpha = 0.6$: precision pari a $\simeq 75\%$ e recall pari a $\simeq 88\%$
    \end{itemize}
    \item senza \texttt{word2keypress}, \texttt{n\_mingram = 2}, \texttt{epoche = 5}
    (figura \ref{quintomodello}):
    \begin{itemize}
        \item $\alpha = 0.5$: precision pari a $\simeq 65\%$ e recall pari a $\simeq 95\%$
        \item $\alpha = 0.6$: precision pari a $\simeq 77\%$ e recall pari a $\simeq 89\%$
    \end{itemize}
\end{itemize}
I modelli con i migliori risultati non utilizzano la libreria \texttt{word2keypress}, e presentano un miglioramento di almeno il 3\% nel valore di precision e con valore di recall invariato, rispetto ai modelli che utilizzano \texttt{word2keypress}.

Un altro importante miglioramento è stato determinato dal valore di \texttt{n\_mingram}: tutti i modelli con \texttt{n\_mingram = 2} hanno avuto un lieve incremento del valore di precision.

Il modello con i risultati più scarsi è quello con \texttt{word2keypress}, \texttt{n\_mingram = 1}, \texttt{epoche = 5} (modello di Bijeeta et
alii~\cite{bijeeta}, figura \ref{primomodello}), e lo si può osservare dai valori inferiori di precision rispetto ad altri modelli.

Il modello con le migliori performance è invece riportato in figura \ref{quintomodello}, senza \texttt{word2keypress}, \texttt{n\_mingram = 2}, \texttt{epoche = 5}, con valore di precision pari a 77\%, per $\alpha = 0.6$, superiore di $\simeq 10\%$ rispetto al modello di Bijeeta et alii~\cite{bijeeta}.

\subsection{Criticità del modello proposto da Bijeeta et alii}
\label{sec:criticita bijeeta}
I motivi per cui il modello di Bijeeta et alii~\cite{bijeeta} è risultato il peggiore sono i seguenti:
\begin{itemize}
    \item La libreria \texttt{word2keypress} traduce ciascun carattere come sequenza di tasti premuti, come accennato nel paragrafo \ref{sec:allenamento pass2path}. Di conseguenza, quando si hanno due password che presentano un'alternanza di caratteri minuscoli e maiuscoli, esse risulteranno molto simili tra loro, a causa della ripetizione delle sequenze \texttt{<s> <c>}.
    \item Sempre a causa di \texttt{word2keypress}, la presenza di caratteri speciali non consecutivi che si ottengono premendo \texttt{SHIFT}, ad esempio:
    \begin{itemize}
        \item la password \texttt{\$1mp@t1c*}, che viene tradotta come \texttt{<s>41mp<s>2t1c<s>8})
        \item la password \texttt{\#wlng\%p*m\}} che viene tradotta come\\ \texttt{<s>3wlng<s>5p<s>8m<s>[}
    \end{itemize}
    rende la valutazione di similarità tra due password poco affidabile, poiché verrebbero valutate come simili, quando in realtà dovrebbero essere valutate come sicure e diverse tra loro.
    \item L'utilizzo di \texttt{n\_mingram = 1} rende la valutazione meno precisa rispetto allo stesso modello con \texttt{n\_mingram = 2}, poiché risulta più efficace valutare la vicinanza di due specifici bigrammi, anziché avere una valutazione maggiormente dispersiva in cui si considera la posizione di un carattere rispetto a un altro. Nelle password, infatti, risulta che i singoli caratteri non dipendano da un insieme di regole (come nel caso della letteratura inglese), ma vengano fortemente influenzati da fattori distinti tra loro e difficilmente rappresentabili.
\end{itemize}

\subsection{Soglia di similarità per la valutazione di due password}
\label{sec:soglia similarita valutazione due password}
Per potere determinare se due password siano simili tra loro occorre definire la soglia di similarità $\alpha$, tenendo in considerazione i valori di precision e recall ottimali.

Nel caso in studio, è necessario avere un valore di recall più alto rispetto al valore di precision, dato che è importante rilevare più password simili tra loro possibili. Questo tuttavia, per valori di precision molti bassi (come accennato nel paragrafo \ref{sec:classificazione password}) può comportare un'alta percentuale di alti positivi, ovvero di coppie di password considerate come simili, anche se abbastanza diverse tra loro.

Un buon compromesso è stato raggiunto con $\alpha = 0.6$ (rappresentato dalle ascisse dei grafici con precision e recall); nel modello mostrato in figura \ref{quintomodello} si ha, per $\alpha = 0.6$ precision pari a $\simeq 77\%$ e recall pari a $\simeq 89\%$.

Nel paper di Bijeeta et alii \cite{bijeeta} è stato scelto $\alpha = 0.5$, tuttavia, come si può vedere in figura \ref{precision recall}, ciò comportava un valore di precision basso, con una variazione di recall pari a $\simeq 5\%$ e di precision pari a $\simeq 10\%$ rispetto ai valori registrati con $\alpha = 0.6$.
\section{Rappresentazione grafica della distanza tra parole}
\label{sec:rappresentazione grafica distanza tra parole}
In questo progetto è stato scelto, per facilitare la comprensione, di rappresentare le similarità tra password mediante un grafico a 3 dimensioni. A questo scopo è stato utilizzato un algoritmo noto come t-SNE\footnote{\url{https://scikit-learn.org/stable/modules/generated/sklearn.manifold.TSNE.html}} per ridurre la dimensione del modello da 200 a 3.
Date due password come \texttt{ipwnedyou} e \texttt{numBerOne} si possono vedere per ciascuna di esse le 5 password più simili proposte dal modello e la distanza in termini di similarità rispetto alla password originaria.
\begin{figure}[H]
    \centering
    \includegraphics[width=15cm]{./immagini/3dplot.png}
    \caption{Grafico 3D per mostrare la similarità tra password}
    \label{3d}
\end{figure}

% Conclusioni
\chapter*{Conclusioni}
\chaptermark{Conclusioni}
\addcontentsline{toc}{chapter}{Conclusioni}
Conclusione.

% Ringraziamenti
\chapter*{Ringraziamenti}
\chaptermark{Ringraziamenti}
\addcontentsline{toc}{chapter}{Ringraziamenti}
Ringrazio chi mi è stato vicino in questo percorso e ha creduto in me. A loro è dedicata questa tesi.

% Bibliografia
%\input{bibliografia/bibliografia}
\phantomsection
\printbibliography[heading=bibintoc]

\end{document}
